\section{FAQs}\label{sec:faq}

\textbf{Why separate the rotation and translation parts while transforming the datapoints?}

We don't want to affect the progress values while performing the transformations.
Particularly,
\begin{equation}
    \begin{aligned}
        {}^{0}\boldsymbol{x}_{n, i}
        =
        \begin{bmatrix}
            {}^{0}\boldsymbol{\xi}_{n, i} \\
            {}^{0}\varphi_{n, i}
        \end{bmatrix}
        \implies
        {}^{m}\boldsymbol{x}_{n, i}
         & =
        \begin{bmatrix}
            {}^{m}\boldsymbol{t} \\
            0
        \end{bmatrix}
        +
        {}^{m}\boldsymbol{H} \; {}^{0}\boldsymbol{x}_{n, i}
        =
        \begin{bmatrix}
            {}^{m}\boldsymbol{t} \\
            0
        \end{bmatrix}
        +
        \begin{bmatrix}
            {}^{m}\boldsymbol{R} & \boldsymbol{0} \\
            \boldsymbol{0}       & 1
        \end{bmatrix}
        \begin{bmatrix}
            {}^{0}\boldsymbol{\xi}_{n, i} \\
            {}^{0}\varphi_{n, i}
        \end{bmatrix}
        \\
        \implies
        \begin{bmatrix}
            {}^{m}\boldsymbol{\xi}_{n, i} \\
            {}^{m}\varphi_{n, i}
        \end{bmatrix}
         & =
        \begin{bmatrix}
            {}^{m}\boldsymbol{t} \\
            0
        \end{bmatrix}
        +
        \begin{bmatrix}
            {}^{m}\boldsymbol{R} \ {}^{0}\boldsymbol{\xi}_{n, i} \\
            {}^{0}\varphi_{n, i}
        \end{bmatrix}
        =
        \begin{bmatrix}
            {}^{m}\boldsymbol{t} + {}^{m}\boldsymbol{R} \ {}^{0}\boldsymbol{\xi}_{n, i} \\
            {}^{0}\varphi_{n, i}
        \end{bmatrix}
    \end{aligned}
\end{equation}
\begin{equation}
    \implies
    {}^{m}\boldsymbol{\xi}_{n, i}
    =
    {}^{m}\boldsymbol{t} + {}^{m}\boldsymbol{R} \ {}^{0}\boldsymbol{\xi}_{n, i}
    \quad \text{and} \quad
    \boxed{
    {}^{m}\varphi_{n, i}
                =
                {}^{0}\varphi_{n, i}
        }
\end{equation}
where implicitly \( {}^{0}\boldsymbol{\xi}_{n, i}, {}^{m}\boldsymbol{R}, {}^{m}\boldsymbol{t}, {}^{m}\boldsymbol{\xi}_{n, i} \) are conformable matrices.
\qed{}

\vspace{1em}
\textbf{Why use the median to find keypoint progress values?}

Median is less sensitive to outliers than the mean, and more robust in general.
What we're trying to achieve is a central progress value for each of the index points; posing this as an optimisation problem, the median estimator is
\begin{equation}
    \mu_{\text{median}} = \arg \min_{a} \mathbb{E} \Big[ \big \vert X - a \big \vert \Big]
\end{equation}
which tries to minimise the expected absolute deviation.

\vspace{1em}
\textbf{Why use variational approximation?}

\vspace{1em}
\textbf{What is \( n_d \)?}

Probably the number of datapoints in the dataset, i.e., \( l \), i.e., 1000.
